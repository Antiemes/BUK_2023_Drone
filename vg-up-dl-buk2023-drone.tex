\documentclass[conference]{IEEEtran}
\IEEEoverridecommandlockouts
% The preceding line is only needed to identify funding in the first footnote. If that is unneeded, please comment it out.
\usepackage{cite}
\usepackage{amsmath,amssymb,amsfonts}
\usepackage{algorithmic}
\usepackage{graphicx}
\usepackage{textcomp}
\usepackage{xcolor}
%\usepackage{booktabs}
\usepackage{tabularray}
\usepackage{flushend}
\UseTblrLibrary{booktabs}
\def\BibTeX{{\rm B\kern-.05em{\sc i\kern-.025em b}\kern-.08em
    T\kern-.1667em\lower.7ex\hbox{E}\kern-.125emX}}
\begin{document}

\title{Data Collection in the Wild: Challenges and Solutions}

\author{\IEEEauthorblockN{Gergely Vakulya}
\IEEEauthorblockA{\small \textit{Alba Regia Technical Faculty} \\
\textit{Óbuda University}\\
\textit{Székesfehérvár, Hungary}\\
\textit{vakulya.gergely@amk.uni-obuda.hu}
}
\and
\IEEEauthorblockN{Péter Udvardy}
\IEEEauthorblockA{\small \textit{Alba Regia Technical Faculty} \\
\textit{Óbuda University}\\
\textit{Székesfehérvár, Hungary}\\
\textit{udvardy.peter@amk.uni-obuda.hu}
}
\and
\IEEEauthorblockN{Levente Dimen}
\IEEEauthorblockA{\small \textit{TODO} \\
\textit{TODO}\\
\textit{TODO}\\
\textit{TODO}
}}

\maketitle

\begin{abstract}
In this paper 
\end{abstract}

\begin{IEEEkeywords}
environment monitoring, data collecion, sensor network, drone technology
\end{IEEEkeywords}

\section{Introduction}
One of the most promising classic sensor networking applications was monitoring
of remote, nonaccessible areas \cite{corke2010}, such as rainforests \cite{wark2008, cama2013}, volcanos \cite{werner2006, song2009} or glaciers \cite{martinez2004, martinez2005}.

\section{Related work}



\section{Summary}

In this paper 

\bibliographystyle{IEEEtran}
\bibliography{IEEEabrv,references}

\end{document}

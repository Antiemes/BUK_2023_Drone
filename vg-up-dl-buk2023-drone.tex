\documentclass[conference]{IEEEtran}
\IEEEoverridecommandlockouts
% The preceding line is only needed to identify funding in the first footnote. If that is unneeded, please comment it out.
\usepackage{cite}
\usepackage{amsmath,amssymb,amsfonts}
\usepackage{algorithmic}
\usepackage{graphicx}
\usepackage{textcomp}
\usepackage{xcolor}
%\usepackage{booktabs}
\usepackage{tabularray}
\usepackage{flushend}
\UseTblrLibrary{booktabs}
\def\BibTeX{{\rm B\kern-.05em{\sc i\kern-.025em b}\kern-.08em
    T\kern-.1667em\lower.7ex\hbox{E}\kern-.125emX}}
\begin{document}

\title{Data Collection in the Wild:\\Challenges and Solutions}

\author{
\IEEEauthorblockN{Péter Udvardy}
\IEEEauthorblockA{\small \textit{Alba Regia Technical Faculty} \\
\textit{Óbuda University}\\
\textit{Székesfehérvár, Hungary}\\
\textit{udvardy.peter@amk.uni-obuda.hu}
}
\and
\IEEEauthorblockN{Levente Dimen}
\IEEEauthorblockA{\small \textit{Department of Cadastre, Civil Engineering} \\
\small \textit{and Environmental Engineering} \\
\textit{1 Decembrie 1918 University}\\
\textit{Alba Iulia, Romania}\\
\textit{dimenlev@yahoo.com}
}
\and
\IEEEauthorblockN{Gergely Vakulya}
\IEEEauthorblockA{\small \textit{Alba Regia Technical Faculty} \\
\textit{Óbuda University}\\
\textit{Székesfehérvár, Hungary}\\
\textit{vakulya.gergely@amk.uni-obuda.hu}
}
}


\maketitle

\begin{abstract}

Extensive monitoring plays a key role in environmental protection. This task, however
has many issues to solve, communication and data collection being a difficult one.
This paper focuses to the challenges of observing hard-to-access areas, e.g.
such as forests and wetlands. The performance of the possible solutions for this problem
are compared, focusing to sensor networking and LPWAN technologies, and a drone based solution
will be proposed. The presented method offers a robust, and reliable, yet simple
data collection solution. The hardware and software architecture, the communication protocol
will be described and the estimated performance of the system is analyzed.


\end{abstract}

\begin{IEEEkeywords}
environment monitoring, data collecion, sensor network, drone technology
\end{IEEEkeywords}

\section{Introduction}

Sensor networks consist of nodes with sensing, processing and communication
capabilities. Their main application area is data collection, especially
on large areas or with large number of points. In most applications
the power sources of the devices are batteries, since the mains current
is not an option. As the continuous lifespan of such a network is
preferrably measured in years, but at least in months, the power
budget must be highly optimized.

One of the most promising classic sensor networking applications was monitoring
of remote, nonaccessible areas \cite{corke2010}, such as rainforests
\cite{wark2008, cama2013}, volcanos \cite{werner2006, song2009} or glaciers
\cite{martinez2004, martinez2005}. 

\section{Related work}

Most conventional sensor network based monitoring
systems use a mesh routing protocol to
send all measurement messages to a dedicated base
station. In a typical setup the sensor nodes have
a limited amount of power (most commonly batteries),
while the power source of the base is practically
unlimited (mains power or solar energy). These protocols
often apply TDMA (Time Division Multiple Access),
which relies on a tight time synchronization.
In such a system each message is relayed through
multiple subsequent nodes, which means extra energy
consumption each time. Another less obvious
disadvantage is, that nodes closer to the base
relay more packets, thus they consume more energy.

Another possible approach is to use one of the
modern LPWAN protocols (e.g. LoRa \cite{erturk2019,
migabo2017} or NBIoT).
First these solutions obviously requite a working
infrastructure. Second they are typically not
designed to transfer large amount of (measurement)
data.

\section{The location of the proposed monitoring system}

\subsection{Wetlands}

Wetlands are usually considered as a specific biotope with high biodiversity
potential located on the edge of the solid surface and the permanent or
seasonal water covered areas. The presence of the water causes saturation into
the soil due to the proximity of the water table. The periodic oxygen-free
status of the ground results in anoxic hydric soils. Wetlands occur in
different forms such as swamps, marches, estuaries, mangrove swamp, marsh,
moorland or peatland.

Wetlands play vital role in biodiversity of vegetation and animal species all
over the world and maintain stable ecosystems. There are many types of water in
wetland soils such as freshwater, seawater or brackish water depending on the
habitat’s location. The vegetation of the wetlands can be woods, trees,
shrubbery, reeds, cattails, or sedges. Climate zones classified by Trewartha
define the type of wetlands.

Beside the importance in habitat protection wetlands can be seen as natural
water filters which collect sediment and pollutants and release almost pure
water to the surrounding areas. Furthermore, wetlands regulate water floodings,
climate change effects and can be considered also as recreational areas and
cultural heritages.

Wetland are threatened by many effects, such as eutrophication, pollution,
sewage and drainage problems, toxic contamination, acidification or
salinisation. These effects are mainly caused by human factors and urban
development.

Wetlands are protected by ‘Ramsar convention on Wetlands of International
Importance Especially as Waterfowl Habitat’ since 1971 in which legal and
institutional framework of the wetland areas’ protection, conservation and
sustainable use was established. The convention defined the list of wetlands of
international importance, for instance Velence and Dinnyés Nature Conservation
Area in Hungary is also on the Ramsar list.

\subsection{The Sóstó natural reserve}

The Sóstó natural reserve is located only two and a half kilometres from the
downtown of Székesfehérvár and can be considered as an organic part of the
city. The wetland is located on 218 hectares of which 121 hectares are
nationally and the rest is locally protected area. The visitor centre is in the
vicinity of the local football arena.

\begin{figure}[htbp]
	\centering
	\includegraphics[width=0.25\textwidth]{fig/map1.png}
  \caption{The map of the Sóstó natural reserve}
	\label{fig-map1}
\end{figure}

In historic times Székesfehérvár was surrounded by deep swamps until the middle
of the 19th century, since then the Sóstó area was partly drained and used as a
recreational area. Two deep wells were used as water supply but after the
Second World War 600 m3 sewage sludge were deposited yearly which caused fast
eutrophication and decay.

The rehabilitation of the Sóstó area was started in the 1990’s and in 2003 the
area got a naturally protected status. New educational trail was formed for
recreational and educational purposes. The Sóstó biotope gives various
possibilities to the flora and fauna, many protected species can be found
there. The water supply of the wetland is provided by local purified wastewater
source, 500 $m^3$ clear water feeds the lakes arrived by a 4 kilometres long
pipeline.


\section{The proposed monitoring method}

\subsection{System architecture}

The architecture of the monitoring system consists of several independent
measuring nodes and a base station, which is attached to a drone, flying
over the measurement area at regular times, allowing to download the
collected data.

The sensor nodes do preprogrammed measurements and store the collected data
in flash memory. 

\begin{figure}[htbp]
	\centering
	\includegraphics[width=0.45\textwidth]{fig/area.png}
  \caption{The architecture of the proposed system}
	\label{fig-area}
\end{figure}

\subsection{The proposed communication protocol}

In the proposed monitoring system the data collector nodes have limited
energy (they are powered by batteries) similarly to regular sensor networking
applications. Using a mesh protocol to collect the measured data each message
would be sent by possibly many sensor nodes. The proposed protocol uses a star
topology instead, which requires to send each message only once, but on the
other hand it assumes direct connection (i.e. line of sight).
The speciality of the system is, that the base station is absent in the huge
majority of the time and present only occasionaly.

The goal of the proposed communication protocol is to detect the presence
of the base station and transfer the collected data when the conditions are
given. The sensor nodes don't necessarily hear each other, therefore collision
avoidance is the task of the base station.

The protocol utilizes the fast acknowledgements of the 802.15.4 MAC layer.
Instead of listening for some kind of beacon messages transmitted by the
base station, the sensor nodes emit \emph{hello} messages
regularly and wait for an \emph{ACK} message.
The time interval of the \emph{hello} messages is a design parameter of
the protocol.

The 802.15.4 ACK packet has 3 unused bits (see Fig. \ref{fig-ack-packet}), which
are used to sign, when the actual data transfer process is possible.
Each of the 8 combinations can sign different waiting times. This way
the base station can spread the time slots of the nodes. The nodes turn off
their radio during waiting. Note that the nodes do not need to maintain tight
time synchronization, for this task.

\begin{figure}[htbp]
	\centering
	\includegraphics[width=0.5\textwidth]{fig/ack-packet.png}
  \caption{The format of the 802.15.4 ACK packet with the 3 reserved bits highlighted
  in the frame control field (FCF)}
	\label{fig-ack-packet}
\end{figure}

During the actual data transfer each node dend send the
previously collected data divided into packets. Each packet is sent
tightly next to each other, with acknowledgements by the base station.


%timeslots for the sensor nodes by the gateway. The 3 bits
%can define 8 possible combinations, where the \emph{000} means an extended
%waiting time. Nodes received that bit combination will send another \emph{hello}
%message, when the 



\subsection{Sensor nodes}

The hardware of the sensor node is based on an Unicomp UCMote Proton B mote.
The base board has an 8-bit ATmega128RFA1 Soc as the main controller. The
microcontroller core is running at 16 MHz, with an approx. 4 mA current
consumption from 3.3 V. The microcontroller has different sleep modes,
and can be stopped from program execution. When stopped, the current consumtion
is reduced to the TODO range. Another important parameter is the wakeup time,
which is in the range of microseconds. The SoC contains a 2.4 GHz radio unit,
that supports the 802.15.4 range standard for low-power low-range communication.
This communication channel is used only during testing for debug purposes only,
for two reasons. First the the 2.4 GHz band is not suitable for the proposed
monitoring system because of the high absorption of. Second, the board has
only a low gain chip antenna connected of the Soc.

The main board contains an additional AT86RF212 radio chip, which operates in
the 868 MHz ISM band. This radio chip can be configured to use different
communication speeds from 20 to 1000 kbps according to the link conditions.
Depending on the communication bandwidth the receiver sensitivity can be as good as
-110 dBm. With the +10dBm maximum power output the achievable link budget can
reach 120 dB, which is well suits to the proposed system. This radio is connected to
an MMCX RF connector, which allows to connect a high gain external antenna.

Since the drone is located above the sensor nodes during the data download
process the usual quarter wave vertical antennas are not a good choice for
this purpose, given they have very low gain in axial direction. QFH antennas \cite{adams1974},
however have good overhead gain, therefore they are much more suitable for the
proposed system.

The power source of the proposed sensor node consists of 4 D type non-rechargeable
lithium batteries, providing 76000 mAh capacity at 3.6 V.

\begin{figure}[htbp]
	\centering
	\includegraphics[width=0.35\textwidth]{fig/ucproton.png}
  \caption{The UCMote Proton B mote.}
	\label{fig-proton}
\end{figure}

The software of the sensor nodes is built around the TinyOS. It is a component
based real-time operating system, especially tailored for low-power
applications. It has support for different types of microcontrollers, radios
and sensors and the uniform interfaces make adding new device drivers and
communication protocols possible.

TinyOS already supported the ATmega128RFA1 SoC with its microcontroller core
and automatically put it in low-power mode, when it has no task to run and no
event to process. The 2.4 GHz radio is also supported and 802.15.4 standard
packets can be sent and received with a simple CSMA/CA MAC protocol. The other
radio chip (AT86RF212, 868 MHz) is also supported with the standard CSMA/CA
protocol, although the special low-power mode had to be implemented. To achieve this,
the header, which contains the source address, must be intercepted and the
time-critical background lookup must be processed in parallel with the recetion
of the remainig part if the packet, before the full packet arrives \cite{vakulya2013}.
Based on
the result of the lookup one of the reserved bits of the acknowledgement packet
is set of cleared, signaling the presence or the absence of a pending packet.

\begin{figure}[htbp]
	\centering
	\includegraphics[width=0.35\textwidth]{fig/protocol.png}
  \caption{The timing diagram of the ACK piggybacking protocol}
	\label{fig-piggyback}
\end{figure}

\section{Drone}

The Chinese DJI company can be considered as one of the leading UAV
manufacturers in the world. The DJI Phantom 3 Advanced UAV is available since
2015 and beside its good features cannot be considered as the latest techniques
\cite{udvardy2019}.

The UAV weighs 1280 grams and has a 15.2V 4480mAh battery which allows maximum
23 minutes’ flight time. The onboard Sony EXMOR 1/2.3” camera has 12 million
pixels and has a FOV of 94° with 20 millimeters focal length. The ISO value
changes between 100 and 3200 for video recording and 100 and 1600 for
photographs. The shutter speed varies between 8 and 1/8000 seconds. The
resolution of the video caption is 2.7K (30fps).

The camera is hanged on-board with the help of a 3-axis gimbal which helps to
stabilize the camera during flight. 

For navigation DJI Phantom 3 UAV uses GPS/GLONASS system outdoor and has a
vision positioning system for indoor positioning. The hover’s horizontal
accuracy is between 0.3 and 1.5 meters and the vertical accuracy is between 0.1
and 0.5 meters depending on the positioning method.

This UAV can reach approximately 3 kilometers distance outdoor from the home
point which is marked before the flight starts. The ‘Go home’ function helps to
find the hover in case of being lost or invisible.

\section{Measurement scenarios}

\subsection{Low density environmental data collection}

In this scenario slowly changing environmental parameters are monitored. The sensor modalities are:

\begin{itemize}

  \item{Soil temperature with 16 bit resolution, one measurement every 5 minutes}
  \item{Air temperature with 16 bit resolution, one measurement every 5 minutes}
  \item{Ambient light intensity with 16 bit resolution, one measurement every 5 minutes}
  \item{Relative humidity with 8 bit resolution, hourly}
  \item{Soil moisture with 8 bit resolution, hourly}

\end{itemize}

A total of 15 sensor nodes are installed in the system.
In each node a total number of 1776 bytes are generated each day.
If the data is collected every two months, approx.
100 kBytes of data is fetched from each node. With a
pessimistic 20 kbps bandwidth approx 1 minute is
necessary to download this data amount from each node.
A 1 minute beacon interval for the sensor nodes would
give a good balance between latency and power
consumption. With these parameters the average flight
time can be estimated to approx. 20 minutes.

The 802.15.4 packet has a 13 byte header (including the preamble,
physical and MAC headers) and a 2 byte footer. With another
The length of a \emph{hello} message with a 5 byte payload is
20 bytes, which requires 8 ms to send with 20 kbps. The length of
the ACK packet is 11 bytes, which requires 4.4 ms of time.
Another approx. 1
ms is required to turn on and off the radio and approx. 2 ms
of gap time is required between the two packets. This adds up
to 15.4 ms. 
The actual data transfer requires approx. 60 seconds of time.
The on time of the radio can be calculated as follows for a year:

\begin{equation}
    T_{on,annual} = 60 \cdot 6 + 365 \cdot 24 \cdot 60 \cdot \frac{15.4}{1000} = 8454 [s]
\end{equation}

Now the annual power consumption of the radio can be calculated as
follows:

\begin{align}
    C_{annual} &= T_{on,annual} * I_{radio,on} = \\
     &= \frac{8454 [s]}{3600} \cdot 20 [mA] = 47 [mAh]
\end{align}

Considering only the radio communication one D type cell would last for
several years (practically the self-discharge limits the lifespan).

\subsection{Pollution related monitoring}

In this scenario, the focus is on pollution related monitoring, specifically
air quality parameters. The sensor modalities and data
collection parameters are as follows:

\begin{itemize}
	\item{Particulate Matter (PM2.5) concentration with 16-bit resolution, one measurement every 1 minute.}
	\item{Carbon Monoxide (CO) concentration with 12-bit resolution, one measurement every 5 minutes.}
	\item{Nitrogen Dioxide (NO2) concentration with 12-bit resolution, one measurement every 5 minutes.}
	\item{Ozone (O3) concentration with 12-bit resolution, one measurement every 5 minutes.}
	\item{Ambient Temperature with 10-bit resolution, one measurement every 1 minute.}
	\item{Relative Humidity with 10-bit resolution, one measurement every 1 minute.}
\end{itemize}

\subsection{Fine grain seismic monitoring}

In this scenario a fine grain environmental parameter (seismic activity) is. Here the
sensor modality is a triaxial accelerometer with 16 bit resolution and 12.5 Hz
sample rate. A total number of 6 nodes are required for this experiment.

This measurement setup generates much more data, than the previous example. Let's
use a one week interval for the data collection. The total data amount during
a week is 2.1 MB in each node. In this case a higher data rate (250 kbps) is more
suitable, which is only achievable when the distance between the nodes and the
base station is smaller (the altitude of the drone is lower) and when no big
obstacles (i.e. large trees) are present.

We can here choose similar beacon interval to the previous example (1 minute). Note
that the nodes can use different data rates for the beaconing and for the
data transfer. The calculated data transfer time is 472 seconds. Based on
the calculations the annual on-time is:

\begin{equation}
    T_{on,annual} = 52 \cdot 472 + 365 \cdot 24 \cdot 60 \cdot \frac{15.4}{1000} = 510198 [s]
\end{equation}

Now the annual power consumption of the radio can be calculated as
follows:

\begin{align}
    C_{annual} &= T_{on,annual} * I_{radio,on} = \\
     &= \frac{510198 [s]}{3600} \cdot 20 [mA] = 1822 [mAh]
\end{align}

Considering a 19000 mAh battery capacity the lifespan of the system would
be approx. 10 years. Note that only the radio communication was taken into account.

\section{Legal background}

\begin{figure*}[t]
	\centering
	\includegraphics[width=0.85\textwidth]{fig/legal.png}
  \caption{}
	\label{fig-piggyback}
\end{figure*}

In Hungary the UAV (unmanned aerial vehicle) and UAS (unmanned aircraft system)
flight is regulated by strict law which is harmonised with the European
legislations (EU Regulation 2019/947 on the rules and procedures for the
operation of unmanned aircraft). https://eur-lex.europa.eu/

Each UAS of which the weight exceeds 120 grams or has onboard data recording
equipment (i.e. camera) must be registered online first. In the following step
the UAS operators also must be registered together with their valid third-party
liability insurance linked to their UAS at the Traffic Authority. The insurance
must cover between 12-15 thousand EURO damage for a year. http://www.legter.hu

UAS operators must get the so-called flight licence and as a drone operator
they get a unique registration number which is valid in all EASA (European
Union Aviation Safety Agency) member state. This flight open category licence
is available for those UAS’s which are less than 25 kilograms of MTOM (maximum
take-off mass), the maximum flight height is 120 meters from the AGL (above
ground level), the UAS must remain in VLOS (visual line of sight) and the UAS
cannot fly over assembly people and cannot transport dangerous materials and
cannot spread any materials to the ground. This category contains A1-A3 and A2
subcategories with the difference in UAS weight, the minimum distance from and
overfly time of uninvolved persons and the horizontal flight speed.
http://www.doe.hu

The competence exam is available and valid in every EU states. In Hungary the
Austrian A1-A3 exam is very popular as it is online and free.

After the fulfilment of the legislative background the GoodID application for
identification purposes and MyDroneSpace ny Hungarocontrol application right
before flight must be used. UAS operators must ask for airspace usage
permission for the exact time. The official administrative deadline for the
permission is 30 days before flight but for special stakeholders there is a
simplified procedure with 5 days lead time.

\section{Summary}

Monitoring physical or natural parameters in harsh environments (e.g. forests and wetlands) has several
challenges, getting the collected data being one of the hardest one. In this paper a novel data
collection approach was presented, using done technology.
The system design with the sensor nodes' hardware, software,
the communication protocol and the drone technology was presented and
calculations were shown for scenarios with different data density.
Calculations support the applicability of the proposed system
in real scenarios.

\bibliographystyle{IEEEtran}
\bibliography{IEEEabrv,references}

\end{document}

\documentclass[conference]{IEEEtran}
\IEEEoverridecommandlockouts
% The preceding line is only needed to identify funding in the first footnote. If that is unneeded, please comment it out.
\usepackage{cite}
\usepackage{amsmath,amssymb,amsfonts}
\usepackage{algorithmic}
\usepackage{graphicx}
\usepackage{textcomp}
\usepackage{xcolor}
%\usepackage{booktabs}
\usepackage{tabularray}
\usepackage{flushend}
\UseTblrLibrary{booktabs}
\def\BibTeX{{\rm B\kern-.05em{\sc i\kern-.025em b}\kern-.08em
    T\kern-.1667em\lower.7ex\hbox{E}\kern-.125emX}}
\begin{document}

\title{Data Collection in the Wild: Challenges and Solutions}

\author{
\IEEEauthorblockN{Péter Udvardy}
\IEEEauthorblockA{\small \textit{Alba Regia Technical Faculty} \\
\textit{Óbuda University}\\
\textit{Székesfehérvár, Hungary}\\
\textit{udvardy.peter@amk.uni-obuda.hu}
}
\and
\IEEEauthorblockN{Levente Dimen}
\IEEEauthorblockA{\small \textit{TODO} \\
\textit{TODO}\\
\textit{TODO}\\
\textit{TODO}
}
\and
\IEEEauthorblockN{Gergely Vakulya}
\IEEEauthorblockA{\small \textit{Alba Regia Technical Faculty} \\
\textit{Óbuda University}\\
\textit{Székesfehérvár, Hungary}\\
\textit{vakulya.gergely@amk.uni-obuda.hu}
}
}

\maketitle

\begin{abstract}

Extensive monitoring plays a key role in environmental protection. This task, however
has many issues to solve, communication and data collection being a difficult one.
This paper focuses to the challenges of observing hard-to-access areas, e.g.
such as forests and wetlands. The performance of the possible solutions for this problem
are compared, focusing to sensor networking and LPWAN technologies, and a drone based solution
will be proposed. The presented method offers a robust, and reliable, yet simple
data collection solution. The hardware and software architecture, the communication protocol
will be described and the estimated performance of the system is analyzed.


\end{abstract}

\begin{IEEEkeywords}
environment monitoring, data collecion, sensor network, drone technology
\end{IEEEkeywords}

\section{Introduction}

Sensor networks consist of nodes with sensing, processing and communication
capabilities. Their main application area is data collection, especially
on large areas or with large number of points. In most applications
the power sources of the devices are batteries, since the mains current
is not an option. As the continuous lifespan of such a network is
preferrably measured in years, but at least in months, the power
budget must be highly optimized.

One of the most promising classic sensor networking applications was monitoring
of remote, nonaccessible areas \cite{corke2010}, such as rainforests
\cite{wark2008, cama2013}, volcanos \cite{werner2006, song2009} or glaciers \cite{martinez2004, martinez2005}. 

\section{Related work}

\section{The proposed monitoring method}

\subsection{Sensor nodes}

The hardware of the sensor node is based on an Unicomp UCMote Proton B mote.

The software of the sensor nodes is built around the TinyOS. It is a component
based real-time operating system, especially tailored for low-power applications.
It has support for different types of microcontrollers, radios and sensors and
the uniform interfaces make adding new device drivers and communication protocols possible.

TinyOS already supported the ATmega128RFA1 SoC with its microcontroller core and
automatically put it in low-power mode, when it has no task to run and no event
to process. The 2.4 GHz radio is also supported and 802.15.4 standard packets can
be sent and received with a simple CSMA/CA MAC protocol. The other radio chip
(ATRF212, 868 MHz) is also supported with the standard CSMA/CA protocol, although
the special low-power mode had to be implemented. To this, the header, which contains
the source address, must be intercepted and the time-critical background lookup must be processed in
parallel with the recetion of the remainig part if the packet, before the full packet arrives.
Based on the result of the lookup one of the reserved bits of the acknowledgement
packet is set of cleared, signaling the presence or the absence of a pending packet.

\section{Summary}

In this paper 

\bibliographystyle{IEEEtran}
\bibliography{IEEEabrv,references}

\end{document}
